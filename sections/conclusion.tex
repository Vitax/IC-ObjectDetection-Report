\begin{document}
During this project I trained multiple Tensorflow models and wrote an application which utilizes a trained model through Tensorflow and is
able to detect objects in an image, video or web camera stream fairly well. Comparing the object detection results of Faster RCNN and SSD,
it is noticeable that the F-RCNN algorithm is performing better in regards to accuracy but is lacks in speed. The reason for that as mentioned during this report, is the complex
structure of the neural network. It performs each task per layer resulting in a slower Model. The SSD algorithm is performing the whole step in one go
and therefore is faster but Less accurate. The results and performances of the trained models vary and further
research for improving these need to be done. Possible feature extensions for the application could be having multiple models set up and
active which could be used for comparing and running analyzes of the performances of different models.\\\\
To sum it up, object detection has evolved and probably will evolve further in the future. We can see at its current state that it is able
to perform complex object detection tasks in a pretty decent way which is close to real time. There are still problems such as the
complexity and variety of images in the real life scenarios. But these types of problems may become less relevant as the amount
of data stored in the Internet and on servers world wide, will increase and new technologies will be developed. Maybe quantum computers
will bring the next breakthrough when modeling new object detection systems and technologies. There are still a lot of ways to improve and
expand on the concept. How fast and how good it will evolve is uncertain.
\end{document}

